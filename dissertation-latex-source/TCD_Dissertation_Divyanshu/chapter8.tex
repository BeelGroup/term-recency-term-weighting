\chapter{Conclusion and Future Work}
\section{Conclusion}
In this research, we proposed a novel algorithm for term weighting. The presented approach shows the significance of temporal distribution along with existing space distribution of terms. We suggest the scheming of a term recency parameter based on the origin of the word and the usage in the document corpus. This factor is used along with the standard weighting values (such as TF and IDF) for relevance scoring in the information retrieval system. The time normalized algorithm suggested is a robust model, that can fit into multiple use cases. The origin year used in the term age calculation can be traced to various sources as per the problem statement. The algorithm is tested on a news dataset, with queries trying to find links with the documents, Web answer retrieval dataset and research papers citations dataset. We implemented this on top of standard term weighting methodologies, TF-IDF, BM25 and Universal Sentence Encoder models.

Experiments conducted on the search relevance system show that term-recency based TF-IDF and tUSE model outperforms the classic TF-IDF and classic USE algorithms with a significant margin when measured in terms of average precision, recall, F1 and NDCG. 
With these significant results, this can be extended to other term weighting methodologies as well. It sets a premise which can be experimented further to several other fields such as user modelling, personalization, recommendations, classification, etc.

\section{Limitations and Future Work}
One of the major drawback of our work was the time-based BM25 model, did not perform well for any of the datasets considered. Time-based BM25 model needs to be analysed in depth to get a better understanding of the limitation and study the introduction of term recency for such models. The possible reason might be term recency is not a valid parameter for BM25 model or the normalization method does not align well with the existing metrics used or we need to check up another normalization or scale. These methods need to be carefully studied to justify the introduction of the term recency parameter in BM25 term weighting.

Another drawback we saw in our existing study was tUSE model did not perform well for the WebAP dataset. We have not done much analysis on the reason for this drop of performance but certain inferences are derived based on the study of the dataset and algorithm. One of the possible reasons for this might be the size of the corpus used, that is, TREC news corpus has approximately 600k documents while Web AP dataset has just 6k documents, which is 100 times less than the former dataset. However, this is an inference based on the results retrieved and has not been verified. There might be other possible reasons, such as the size of documents, size of queries used, number of proper nouns in the queries, etc. Or probably term age might not be a relevant metric for this dataset. These possible reasons still need to be analyzed before affirming out a conclusion on these contrasting results. This part can also be reviewed for a short study and understanding the working of term age-based USE model.

As per our current model, we have introduced an elastic plugin to include the term age parameter indexed as payload in the Elasticsearch indices. However, this way of calculating relevancy scores and fetching results is not a very efficient method of information retrieval. For obvious reasons of adding extra payload and computation, this method retrieves results much slower than the standard way of information retrieval through Elasticsearch. An effective way to reduce latency in fetching the relevant results can also be taken up as a part of the future scope for this project.


Another important experiment that can be carried out in the future work is to test out different time normalization and time scaling parameters for calculating term age. We have just tested one normalization based on retrieving term-age as documents per year, that gave out significant results for our tested datasets. These can be extended to study different time normalization factors and there impact on the results fetched. Also, the use of term age parameter in the equation for term weight, like the impact of dividing the factor, or adding some constant to normalize the terms. And a different level of term age, like squaring, square root, etc. can be tested for further analysis on the topic.


We performed our study mainly focused on information retrieval task, however, there are many other applications for term weighting, where term recency can be applied. Term age parameter can be introduced in other tasks such as text classification, user modelling, recommender systems and other related text-based models. Term recency has an ample scope in user modelling space, by building out user models based on the short term contexts and long term contexts. And further using these models for recommendations or improving search results. At this point, this study can be considered as a strong premise for the validation of the term recency model in the term weighting methodology and can be extended for a vast scope.


